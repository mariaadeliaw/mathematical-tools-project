% Options for packages loaded elsewhere
\PassOptionsToPackage{unicode}{hyperref}
\PassOptionsToPackage{hyphens}{url}
%
\documentclass[
]{article}
\usepackage{amsmath,amssymb}
\usepackage{iftex}
\ifPDFTeX
  \usepackage[T1]{fontenc}
  \usepackage[utf8]{inputenc}
  \usepackage{textcomp} % provide euro and other symbols
\else % if luatex or xetex
  \usepackage{unicode-math} % this also loads fontspec
  \defaultfontfeatures{Scale=MatchLowercase}
  \defaultfontfeatures[\rmfamily]{Ligatures=TeX,Scale=1}
\fi
\usepackage{lmodern}
\ifPDFTeX\else
  % xetex/luatex font selection
\fi
% Use upquote if available, for straight quotes in verbatim environments
\IfFileExists{upquote.sty}{\usepackage{upquote}}{}
\IfFileExists{microtype.sty}{% use microtype if available
  \usepackage[]{microtype}
  \UseMicrotypeSet[protrusion]{basicmath} % disable protrusion for tt fonts
}{}
\makeatletter
\@ifundefined{KOMAClassName}{% if non-KOMA class
  \IfFileExists{parskip.sty}{%
    \usepackage{parskip}
  }{% else
    \setlength{\parindent}{0pt}
    \setlength{\parskip}{6pt plus 2pt minus 1pt}}
}{% if KOMA class
  \KOMAoptions{parskip=half}}
\makeatother
\usepackage{xcolor}
\usepackage[margin=1in]{geometry}
\usepackage{graphicx}
\makeatletter
\def\maxwidth{\ifdim\Gin@nat@width>\linewidth\linewidth\else\Gin@nat@width\fi}
\def\maxheight{\ifdim\Gin@nat@height>\textheight\textheight\else\Gin@nat@height\fi}
\makeatother
% Scale images if necessary, so that they will not overflow the page
% margins by default, and it is still possible to overwrite the defaults
% using explicit options in \includegraphics[width, height, ...]{}
\setkeys{Gin}{width=\maxwidth,height=\maxheight,keepaspectratio}
% Set default figure placement to htbp
\makeatletter
\def\fps@figure{htbp}
\makeatother
\setlength{\emergencystretch}{3em} % prevent overfull lines
\providecommand{\tightlist}{%
  \setlength{\itemsep}{0pt}\setlength{\parskip}{0pt}}
\setcounter{secnumdepth}{-\maxdimen} % remove section numbering
\ifLuaTeX
  \usepackage{selnolig}  % disable illegal ligatures
\fi
\usepackage{bookmark}
\IfFileExists{xurl.sty}{\usepackage{xurl}}{} % add URL line breaks if available
\urlstyle{same}
\hypersetup{
  pdftitle={Environmental and Behaviour Effects on Lizards Morphological Traits},
  pdfauthor={Maria \& Mathew},
  hidelinks,
  pdfcreator={LaTeX via pandoc}}

\title{Environmental and Behaviour Effects on Lizards Morphological
Traits}
\author{Maria \& Mathew}
\date{2024-12-18}

\begin{document}
\maketitle

\subsection{Background}\label{background}

\subsection{Methods}\label{methods}

From the dataset given by citation, it went through an initial data
cleaning process to retain only the columns relevant for subsequent
analyses. Missing data were addressed through two potential approaches:
deletion or imputation. Given that dropping missing data would result in
an 80\% reduction in dataset size and a loss of variability across
certain variables, we chose data imputation using the MICE algorithm.

To examine relationships among key variables, a correlation matrix was
computed using Pearson's correlation coefficients. Variables were
standardized prior to analysis to ensure comparability. The resulting
correlation matrix was visualized using the \texttt{corrplot} package in
R

Principal Component Analysis (PCA) was applied to the continuous
variables to explore potential clustering patterns. However, as PCA did
not show any distinct clusters, we implemented k-means clustering with
three groups. This approach resulted in apparent patterns within the
dataset. To further investigate these clusters, we performed a
Chi-square test to evaluate their associations with categorical
variables, including foraging mode, preferred habitat type, and mode of
reproduction.

\subsection{Results}\label{results}

The correlation matrix highlights strong relationships among key
variables. Latitude shows a weak correlation with morphological
variables, such as mean snout-vent length (SVL) and weight, while
relative clutch mass (RCM) and mean clutch size are positively
correlated with each other. These patterns suggest ecological or
life-history trade-offs that may influence clustering in subsequent
analyses.

\includegraphics{lizarding_files/figure-latex/unnamed-chunk-4-1.pdf}

The first two principal components explained 43.2 (PC1) and 19.7 (PC2)
percent of the variance, respectively. Variables such as latitude, RCM,
and mean clutch size strongly contributed to PC1. The k-means clustering
identified three distinct groups, visualized in the PCA biplot.

\includegraphics{lizarding_files/figure-latex/unnamed-chunk-5-1.pdf}

Figure 2 PCA bi-plot showing the clustering of individuals into three
groups based on k-means clustering.

\begin{itemize}
\tightlist
\item
  \emph{\textbf{Cluster 1 (Red):} Larger body sizes, lower relative
  clutch mass (RCM) and mean clutch size, and occurrence at lower
  latitudes.}
\item
  \emph{\textbf{Cluster 2 (Green):} Higher RCM and mean clutch size,
  with distribution at higher latitudes.}
\item
  \emph{\textbf{Cluster 3 (Blue):} Smaller body sizes and weights, lower
  RCM and clutch sizes, and presence at lower latitudes.}
\end{itemize}

\begin{verbatim}
## Warning in stats::chisq.test(x, y, ...): Chi-squared approximation may be
## incorrect
\end{verbatim}

\begin{verbatim}
## # A tibble: 4 x 4
##   Variable               X_squared    df      p_value
##   <chr>                      <dbl> <int>        <dbl>
## 1 Preferred Habitat Type    54.0      14 0.00000126  
## 2 Foraging Mode              8.78      2 0.0124      
## 3 Mode of Reproduction      33.3       2 0.0000000593
## 4 Distribution               0.488     2 0.783
\end{verbatim}

The Chi-squared tests revealed significant associations between cluster
groups and various ecological and reproductive traits. These results
indicate that the clusters are strongly influenced by ecological traits,
with distribution showing the strongest association, suggesting distinct
patterns of geographic or habitat specialization among clusters

\subsection{Discussion}\label{discussion}

Lizards species seem to be differentiated by their clutch size and their
SVL. The clustering in Figure \# shows that there are two groups
influenced mainly by variables related to clutch size (green) and SVL
and weight (blue). This can be explained by the adaptive vs plasticity
hypothesis. Here, differences observed can be matched with geographical
adaptation but, this can prove blurry to see when this phenotype changes
come from plasticity
(\href{https://link.springer.com/article/10.1007/s11692-013-9247-2}{citation}).
The effect of plasticity on the measurement of individuals can be
causing noise in the data, that may explain why there is a sole cluster
(green), that is unexplained by the metrics used in this database.
However, this analysis is limited due to the absent of more variables
that could help explain the grouping of the individuals in the green
cluster.

Latitude is driving the the clutch size (including RCM) of lizards
species. Here, latitude is used as a proxy for climate, variable that is
lacking in the database. This can be observed in Figure \#, where the
variance for the green cluster is mostly explained by latitude.
Nonetheless, despite the fact that coloring the graphs with different
factos from the data, by conducting a chi square test, we found out that
this clustering is significantly (\emph{p} \textless{} 0.05) backed up
by habitat type, distribution, foraging mode, and mode of
reproduction.This is consistent with literature that states that
relative clutch mass (RCM) in lizards is influenced by foraging mode,
predator escape tactics, and resource availability, with higher RCM in
sit-and-wait foraging species and lower RCM in widely foraging species
(\href{https://www.notion.so/ECOLOGICAL-AND-EVOLUTIONARY-DETERMINANTS-OF-RELATIVE-CLUTCH-MASS-IN-LIZARDS-Semantic-Scholar-16bfcf8b78b080e4a4c1d8f684686178?pvs=21}{citation}).
However, is also important to consider that RCM in lizards need not be
correlated to reproductive effort, and, if it is, then reproductive
effort co-evolved with predator escape and foraging strategies and
ecologically analogous species should not only exhibit similar RCM
values, but also similar reproductive efforts
(\href{https://www.notion.so/Body-Shape-Reproductive-Effort-and-Relative-Clutch-Mass-in-Lizards-Resolution-of-a-Paradox-Sema-16bfcf8b78b0804094ccf40336e02277?pvs=21}{citation}).
In the case of species with both oviparous and viviparous populations,
viviparous females often have higher RCM due to the need to accommodate
developing embryos
(\href{https://www.notion.so/Maternal-body-volume-as-a-constraint-on-reproductive-output-in-lizards-evidence-from-the-evolution--16bfcf8b78b080c5bdf3f61c4fd06758?pvs=21}{citation}).
Still, as previously mentioned, some other variables are needed in the
PCA to provide a better picture of what are the ecological implications
of this clustering.

\subsection{Conclusion}\label{conclusion}

Latitudinal gradient and mode of reproduction are key drivers of the
morphology of lizards. In this relationship, latitude seems to have a
major influence on the reproduction, rather than in how large the lizard
is. The size of the lizard may be explained by some other ecological
factor that has not been included in this study. Understanding this
ecological patterns can unravel new information that may allow us to
better understand the conservation challenges for this species in front
of a climate change scenario. It is important to include in the methods
a way of measuring the effect of plasticity, so that we can better
distinguish it from adaptive difference when trying to relate this to
environmental factors.

\end{document}
